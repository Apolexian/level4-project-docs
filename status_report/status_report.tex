\documentclass[11pt]{article}
\usepackage{times}
\usepackage{fullpage}
\usepackage[english]{babel}
    
    \title{MQTT over QUIC}
    \author{Ivan Nikitin - 2292523n}

    \begin{document}
    \maketitle
    
    
     

\section{Status report}

\subsection{Proposal}\label{proposal}

\subsubsection{Motivation}\label{motivation}

With the number of devices reaching 12 billion in 2020~\cite{noauthor_state_2020}, IoT is becoming the dominant form of computation.
Despite this, security in IoT devices still leaves much to be desired.
The inherent hardware constraints and challenges in updatability mean that these devices are susceptible to attacks in the areas of data integrity, privacy, eavesdropping, and more~\cite{alaba_internet_2017}.

MQTT is a widely-used message-passing protocol used for IoT devices.
It relies on the transport protocol used in its implementation to provide both reliable delivery and security.
QUIC is a new general-purpose transport layer protocol that is set to improve upon TCP.
By using QUIC as the transport layer protocol for MQTT we can provide the essential security features, mainly authentication and encryption at packet and header level, needed to secure MQTT.

Securing MQTT via QUIC should, however, not come at a performance overhead compared to current implementations.
Performance overheads will therefore be addressed by focusing on three major avenues.

Rust is a new systems programming language with type, memory, and concurrency safety guarantees while not impeding performance.
Evaluating the feasibility of using rust for IoT devices may show that these guarantees do not have to come with substantial overhead, which is crucial for hardware-constrained devices.

Evaluating the approach used to trimm down the QUIC stack as presented by Lars Eggert~\cite{eggert_towards_2020} will be the next avenue for evaluation.
The reproducability of these methods will show if a general approach to porting transport protocols to hardware-constrained devices is feasible. 

Providing a feature-full alternative to TLS, which comes with major hardware overheads, will also be important to make deploying secure implementations of MQTT feasible.
The alternative that will be explored in this project will be key-policy attribute-based encryption (KP-ABE) due to the constant computational cost needed to create a ciphertext with respect to the number of attributes~\cite{wang_key-policy_2012}.

\subsubsection{Aims}\label{aims}

The aims for the project will be presented in the form of a MoSCoW analysis.

\subsection*{Must}

Must port an existing MQTT implementation to use a QUIC implementation.
The implementations chosen will use the Rust programming language as specified in the motivation.
This will be the foundation for all further analysis, hence is required to be completed.\
This port will then be evaluated against existing industry-standard MQTT implementations as well as the MQTT using QUIC implementation in C presented by Puneet Kumar~\cite{kumar_implementation_2019}.
The evaluation will be conducted with respect to binary size, heap and stack allocations as well as latency.

Must analyse authentication and encryption feature fullness of the implementation.
These factors are the major features that provide a way to secure MQTT.
Authentication means that non-trusted devices can not simply subscribe to all topics on a given broker to eavesdrop.
Authentication also limits attack vectors to devices on the same network.
Encryption of packets and headers also prevents various man in the middle attacks, especially at the stage of authenticating with the broker.
In combination, these also guarantee data integrity.

\subsection*{Should}

Should evaluate the methods of thinning down the QUIC stack for hardware constrained devices and the reproducibility of these methods.
As discussed in the motivation this will analyse the feasibility of a general framework to thin down transport protocol implementations for IoT devices.
This analysis will also be the basis to motivate the need for alternatives to TLS for hardware constrained devices due to its major overhead.

Should evaluate the overhead of TLS and present a comparison to KP-ABE.
This is important to present a possible solution to the overhead issue while maintaining the security features needed for a secure MQTT implementation.
It is also a basis for analysing which features of TLS are needed for hardware constrained devices.

\subsection*{Could}

Could extend the QUIC implementation to use a KP-ABE based cypher suite instead of TLS.
Providing such an extension to QUIC would provide an alternative, thin QUIC stack for hardware constrained devices.
This however would take a large amount of effort, hence is not higher on the priority scale and may be left as a future extension depending on time constraints.

\subsection*{Won't have this time}

Consider the performance of the resulting implementation in the context of mesh networks.
QUIC should outperform TCP in congested networks, which leads well to its use in IoT mesh networks.
Similarly, the security of mesh networks is highly topical, especially in the areas of botnets and device recruitment.
However, for the scope of this project, the set up of a test bench for a mesh network and the evaluation would take too much time.

\subsection{Progress}\label{progress}

\emph{{[}Briefly state your progress so far, as a bulleted list{]}}
s   
\subsection{Problems and risks}\label{problems-and-risks}

\subsubsection{Problems}\label{problems}

\emph{{[}What problems have you had so far, that have held up the
project?{]}}

\subsubsection{Risks}\label{risks}

\emph{{[}What problems do you foresee in the future and how will you
mitigate them?{]}}

\subsection{Plan}\label{plan}

\emph{{[}Time plan, in roughly weekly to monthly blocks, up until
submission week{]}}

\medskip
\bibliographystyle{acm}
\bibliography{ref}    
    

\end{document}
