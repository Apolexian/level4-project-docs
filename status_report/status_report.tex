\documentclass[11pt]{article}
\usepackage{times}
\usepackage{fullpage}
\usepackage[english]{babel}
    
    \title{MQTT over QUIC}
    \author{Ivan Nikitin - 2292523n}

    \begin{document}
    \maketitle
    
    
     

\section{Status report}

\subsection{Proposal}\label{proposal}

\subsubsection{Motivation}\label{motivation}

With the number of devices reaching 12 billion in 2020~\cite{noauthor_state_2020}, IoT is becoming the dominant form of computation.
Despite this, security in IoT devices still leaves much to be desired.
The inherent hardware constraints and challenges in updatability mean that these devices are susceptible to attacks in the areas of data integrity, privacy, eavesdropping, and more~\cite{alaba_internet_2017}.

MQTT is a widely-used message-passing protocol used for IoT devices.
It relies on the transport protocol used in its implementation to provide both reliable delivery and security.
QUIC is a new general-purpose transport layer protocol that is set to improve upon TCP.
By using QUIC as the transport layer protocol for MQTT we can provide the essential security features, mainly authentication and encryption at packet and header level, needed to secure MQTT.

Securing MQTT via QUIC should, however, not come at a performance overhead compared to current implementations.
Performance overheads will therefore be addressed by focusing on three major avenues.

Rust is a new systems programming language with type, memory, and concurrency safety guarantees while not impeding performance.
Evaluating the feasibility of using rust for IoT devices may show that these guarantees do not have to come with substantial overhead, which is crucial for hardware-constrained devices.

Evaluating the approach used to trimm down the QUIC stack as presented by Lars Eggert~\cite{eggert_towards_2020} will be the next avenue for evaluation.
The reproducability of these methods will show if a general approach to porting transport protocols to hardware-constrained devices is feasible. 

Providing a feature-full alternative to TLS, which comes with major hardware overheads, will also be important to make deploying secure implementations of MQTT feasible.
The alternative that will be explored in this project will be key-policy attribute-based encryption (KP-ABE) due to the constant computational cost needed to create a ciphertext with respect to the number of attributes~\cite{wang_key-policy_2012}.

\subsubsection{Aims}\label{aims}

The aims for the project will be presented in the form of a MoSCoW analysis.

\subsubsection*{Must}

Must port an existing MQTT implementation to use a QUIC implementation.
The implementations chosen will use the Rust programming language as specified in the motivation.
This will be the foundation for all further analysis, hence is required to be completed.\
This port will then be evaluated against existing industry-standard MQTT implementations as well as the MQTT using QUIC implementation in C presented by Puneet Kumar~\cite{kumar_implementation_2019}.
The evaluation will be conducted with respect to binary size, heap and stack allocations as well as latency.

Must analyse authentication and encryption feature fullness of the implementation.
These factors are the major features that provide a way to secure MQTT.
Authentication means that non-trusted devices can not simply subscribe to all topics on a given broker to eavesdrop.
Authentication also limits attack vectors to devices on the same network.
Encryption of packets and headers also prevents various man in the middle attacks, especially at the stage of authenticating with the broker.
In combination, these also guarantee data integrity.

\subsubsection*{Should}

Should evaluate the methods of thinning down the QUIC stack for hardware constrained devices and the reproducibility of these methods.
As discussed in the motivation this will analyse the feasibility of a general framework to thin down transport protocol implementations for IoT devices.
This analysis will also be the basis to motivate the need for alternatives to TLS for hardware constrained devices due to its major overhead.

Should evaluate the overhead of TLS and present a comparison to KP-ABE.
This is important to present a possible solution to the overhead issue while maintaining the security features needed for a secure MQTT implementation.
It is also a basis for analysing which features of TLS are needed for hardware constrained devices.

\subsubsection*{Could}

Could extend the QUIC implementation to use a KP-ABE based cypher suite instead of TLS.
Providing such an extension to QUIC would provide an alternative, thin QUIC stack for hardware constrained devices.
This however would take a large amount of effort, hence is not higher on the priority scale and may be left as a future extension depending on time constraints.

\subsubsection*{Won't have this time}

Consider the performance of the resulting implementation in the context of mesh networks.
QUIC should outperform TCP in congested networks, which leads well to its use in IoT mesh networks.
Similarly, the security of mesh networks is highly topical, especially in the areas of botnets and device recruitment.
However, for the scope of this project, the set up of a test bench for a mesh network and the evaluation would take too much time.

\subsection{Progress}\label{progress}

The first semester of the project was spent on a short literature review and planning section, setting up a test-bench using mininet and vagrant, and on the mqtt port.
The test-bench is ready for evaluation with both QUIC and MQTT tested on the vagrant VM.
The MQTT port required a service layer API for an existing QUIC implementation to be created.
Initially, quiche was used for this, however there were complications due to the library not supporting async.
Hence, this was switched to quinn, a tokio based QUIC implementation.
The resulting implementation uses quinn to send and receive QUIC packets and seems to work correctly.
The MQTT port has a rather complicated code base and has proved a little challenging to port using the QUIC implementation.
At the start of writing there are still some bugs in the sending process, however I expect this to be completed over the christmas break.

\subsection{Problems and risks}\label{problems-and-risks}

\subsubsection{Problems}\label{problems}

The problems encountered so far have largely been in the use of the QUIC and MQTT implementations.
The TLS certificate generation example for quinn did not work out of the box and had to be altered.
There are no hard blocking problems, however a lot of the work is an exercise in finding small bugs with little documentation.

\subsubsection{Risks}\label{risks}

Problems in the future can arise from the same sources as this semester.
That is, some of the implementations can again prove challenging, however I expect that this can largely be addressed over the holiday break by dedicating a lot of time to debugging.
Alternatively, in the worst case scenario, a re-scoping to just QUIC has been discussed as the QUIC part of the project already works.
In this case, any time in the following plan that is dedicated to MQTT will be instead be used for further work on the KP-ABE parts of the project.

Any potential problems with getting the evaluation setup on devices has already been circumvented by using a mininet test-bed.

\subsection{Plan}\label{plan}

The holiday break will be spent on fixing the remaining bugs in the MQTT port and getting it to the stage where it can be evaluated.
Additionally, the evaluation of the QUIC implementation and the potential ways to decrease the size of the binary will also be done during this time.
The background research of KP-ABE and existing implementations of libraries for this in rust will also be done during this time.
Week 17 will be spent on the evaluation and analysis of the MQTT implementation.
Weeks 18 to 20 will be spent on analysing the key features of TLS and comparing this to what can be provided by KP-ABE.
Weeks 21 to 23 will be spent on getting the QUIC implementation to work with KP-ABE and analysing the size of teh resulting binaries. 
If this is not managebale to an extent where the analysis is possible then this time will instead be spent analysing what it would take to switch away from TLS for IoT.
Finally, weeks 24 to 27 will be spent on writing the dissertation and getting feedback on it to improve it incrementally.

\medskip
\bibliographystyle{acm}
\bibliography{ref}    
    

\end{document}
